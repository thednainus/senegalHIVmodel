\documentclass[12pt,]{article}
\usepackage{lmodern}
\usepackage{amssymb,amsmath}
\usepackage{ifxetex,ifluatex}
\usepackage{fixltx2e} % provides \textsubscript
\ifnum 0\ifxetex 1\fi\ifluatex 1\fi=0 % if pdftex
  \usepackage[T1]{fontenc}
  \usepackage[utf8]{inputenc}
\else % if luatex or xelatex
  \ifxetex
    \usepackage{mathspec}
  \else
    \usepackage{fontspec}
  \fi
  \defaultfontfeatures{Ligatures=TeX,Scale=MatchLowercase}
\fi
% use upquote if available, for straight quotes in verbatim environments
\IfFileExists{upquote.sty}{\usepackage{upquote}}{}
% use microtype if available
\IfFileExists{microtype.sty}{%
\usepackage{microtype}
\UseMicrotypeSet[protrusion]{basicmath} % disable protrusion for tt fonts
}{}
\usepackage[margin=1in]{geometry}
\usepackage{hyperref}
\hypersetup{unicode=true,
            pdftitle={Senegal HIV model},
            pdfauthor={Fabricia F. Nascimento},
            pdfborder={0 0 0},
            breaklinks=true}
\urlstyle{same}  % don't use monospace font for urls
\usepackage{graphicx,grffile}
\makeatletter
\def\maxwidth{\ifdim\Gin@nat@width>\linewidth\linewidth\else\Gin@nat@width\fi}
\def\maxheight{\ifdim\Gin@nat@height>\textheight\textheight\else\Gin@nat@height\fi}
\makeatother
% Scale images if necessary, so that they will not overflow the page
% margins by default, and it is still possible to overwrite the defaults
% using explicit options in \includegraphics[width, height, ...]{}
\setkeys{Gin}{width=\maxwidth,height=\maxheight,keepaspectratio}
\IfFileExists{parskip.sty}{%
\usepackage{parskip}
}{% else
\setlength{\parindent}{0pt}
\setlength{\parskip}{6pt plus 2pt minus 1pt}
}
\setlength{\emergencystretch}{3em}  % prevent overfull lines
\providecommand{\tightlist}{%
  \setlength{\itemsep}{0pt}\setlength{\parskip}{0pt}}
\setcounter{secnumdepth}{0}
% Redefines (sub)paragraphs to behave more like sections
\ifx\paragraph\undefined\else
\let\oldparagraph\paragraph
\renewcommand{\paragraph}[1]{\oldparagraph{#1}\mbox{}}
\fi
\ifx\subparagraph\undefined\else
\let\oldsubparagraph\subparagraph
\renewcommand{\subparagraph}[1]{\oldsubparagraph{#1}\mbox{}}
\fi

%%% Use protect on footnotes to avoid problems with footnotes in titles
\let\rmarkdownfootnote\footnote%
\def\footnote{\protect\rmarkdownfootnote}

%%% Change title format to be more compact
\usepackage{titling}

% Create subtitle command for use in maketitle
\newcommand{\subtitle}[1]{
  \posttitle{
    \begin{center}\large#1\end{center}
    }
}

\setlength{\droptitle}{-2em}
  \title{Senegal HIV model}
  \pretitle{\vspace{\droptitle}\centering\huge}
  \posttitle{\par}
  \author{Fabricia F. Nascimento}
  \preauthor{\centering\large\emph}
  \postauthor{\par}
  \predate{\centering\large\emph}
  \postdate{\par}
  \date{2018-04-10}

\usepackage{booktabs}
\usepackage{longtable}
\usepackage{array}
\usepackage{multirow}
\usepackage[table]{xcolor}
\usepackage{wrapfig}
\usepackage{float}
\usepackage{colortbl}
\usepackage{pdflscape}
\usepackage{tabu}
\usepackage{threeparttable}
\usepackage[normalem]{ulem}

\begin{document}
\maketitle

\hypertarget{introduction}{%
\section{Introduction}\label{introduction}}

\begin{itemize}
\tightlist
\item
  Phylogenetic trees were estimated for each HIV-1 subtype: B, C, and
  02\_AG;
\item
  Using information on when sequences were collected, a phylogenetic
  tree in which branch length were in units of calendar times were also
  estimated;
\item
  These 3 dated phylogenetic tree were merged into a single tree;
\item
  Each tip of the pylogenetic tree was associated to a state (see
  section below);
\item
  If a tip could not be associated to a state (because of missing
  information in the metadata), this particular tip was removed from the
  dated tree;
\item
  The dated phylogenetic tree in which all tips could be assigned to a
  state were then used with phydynR to estimate the transmission rates
  and parameters of the model (see below for more information on which
  parameters we are estimating).
\end{itemize}

\hypertarget{the-model}{%
\section{The Model}\label{the-model}}

The model we fit is based on the structured coalescent models (Volz
2012). These models are used to estimate epidemiological parameters
using a phylogenetic tree and information on states of each tip of the
tree. These states are discrete-trait information representing each
sequence.

In our mathematical model we have 4 different discrete-traits associated
to each DNA sequence:

\begin{itemize}
\tightlist
\item
  \(gpf\) = infected heterosexual females from the general population;
\item
  \(gpm\) = infected heterosexual males from the general population;
\item
  \(msm\) = infected male that have sex with other men;
\item
  \(src\) = source sample, which are infected individuals that are from
  other countries and not from Senegal.
\end{itemize}

\hypertarget{stage-of-infection}{%
\subsection{Stage of infection}\label{stage-of-infection}}

We fit the HIV epidemic in Senegal using ordinary differential equations
(ODE) and only 1 stage of infection. This means that infected
individuals would die and not recover from the infection. In our model
we represented it as \(\gamma\) rate. We used 1 stage of infection,
because the metadata available for the Senegal sequences did not have
information that we could use to determine the stage of HIV infection at
the time the samples were collected.

\hypertarget{how-transmissions-were-modelled}{%
\subsection{How transmissions were
modelled?}\label{how-transmissions-were-modelled}}

\begin{itemize}
\tightlist
\item
  An infected \(msm\) (\(I_{msm}\)) could transmite to another \(msm\)
  with probability \(p_{msm2msm}\)
\item
  An infected \(msm\) (\(I_{msm}\)) could transmit to a \(gpf\) with
  probability \((1 - p_{msm2msm})\)
\item
  An infected \(gpf\) (\(I_{gpf}\)) could transmit to a \(gpm\) with
  probability \(p_{gpf2gpm}\)
\item
  An infected \(gpf\) (\(I_{gpf}\)) could transmit to a \(msm\) with
  probability \((1 - p_{gpf2gpm})\)
\item
  An infected \(gpm\) (\(I_{gpm}\)) could also transmit to a \(gpf\).
  For this event, we used the risk ratio of a male to transmite to a
  female, and fixed it to \(1.02\). This is the parameter \(male_{x}\)
  of our model.
\end{itemize}

\hypertarget{how-about-hiv-incidence-rate}{%
\subsection{How about HIV incidence
rate?}\label{how-about-hiv-incidence-rate}}

We also modelled the HIV incidence rate as a funtion of time (\(t\)) in
\(msm\) and the \(gp\) (general population) as different spline
functions (Eilers and Marx 1996), that in our ODEs are represented by
\(\lambda(t)\) and \(\mu(t)\), respectively.

\hypertarget{the-source-compartment}{%
\subsection{\texorpdfstring{The \(source\)
compartment}{The source compartment}}\label{the-source-compartment}}

Finally, to model the HIV epidemic in Senegal, we also added an
additional compartment named ``source'' (\(src\)), that represents the
rate in which HIV lineages are imported to Senegal from other countries.
We modelled this as a constant efective population size rate with two
parameters to be estimeted -- \(srcNe\): the effective source population
size; and the \(import\) rate. Because the number of imported HIV
balances the number of exported HIV, the infected \(src\) individuals
along time are not represented in the ODEs.

\hypertarget{the-odes-or-mathematical-model-equations}{%
\subsection{The ODEs or mathematical model
equations}\label{the-odes-or-mathematical-model-equations}}

\(\dot{I}_{gpf} = male_x \mu(t) I_{gpm} + (1 - p_{msm2msm}) \lambda(t) I_{msm} - \gamma I_{gpf}\)

\(\dot{I}_{gpm} = p_{gpf2gpm} \mu(t) I_{gpf} - \gamma I_{gpm}\)

\(\dot{I}_{msm} = (1 - p_{gpf2gpm}) \mu(t) I_{gpf} + p_{msm2msm} \lambda(t) I_{msm} - \gamma I_{msm}\)

\hypertarget{estimation-of-epidemiological-parameters}{%
\section{Estimation of epidemiological
parameters}\label{estimation-of-epidemiological-parameters}}

For the Senegal HIV model, we are estimating the parameters using a
Markov chain Monte Carlo (MCMC) as implemented in the R package
\href{\%22https://github.com/florianhartig/BayesianTools\%22}{BayesianTools}.

\hypertarget{parameters-to-be-estimated-and-priors}{%
\subsection{Parameters to be estimated and
priors}\label{parameters-to-be-estimated-and-priors}}

\textbf{Parameters for estimating the spline function for the
\emph{gp}:}

\begin{itemize}
\tightlist
\item
  \emph{gpsp0}: prior chosen with mean around 1.1
\item
  \emph{gpsp1}: prior chosen with mean around 1.1
\item
  \emph{gpsp2}: prior chosen with mean around 1.1
\item
  \emph{gpsploc}
\end{itemize}

\textbf{Parameters for estimating the spline function for the
\emph{msm}:}

\begin{itemize}
\tightlist
\item
  \emph{msmsp0}: prior chosen with mean around \(R_0 = 1.1\)
\item
  \emph{msmsp1}: prior chosen with mean around \(R_0 = 1.1\)
\item
  \emph{msmsp2}: prior chosen with mean around \(R_0 = 1.1\)
\item
  \emph{msmsploc}
\end{itemize}

\textbf{Parameters that controls the \emph{src}:}

\begin{itemize}
\tightlist
\item
  \emph{import}: prior chosen with mean around \(0.03\)
\item
  \emph{srcNe}: prior chosen with mean around \(0.05\)
\end{itemize}

\textbf{Probability of certain events to occour:}

\begin{itemize}
\tightlist
\item
  \emph{pmsm2msm} : prior chosen with mean around \(0.80\)
\item
  \emph{pgpf2gpm} : prior chosen with mean around \(0.80\)
\end{itemize}

\textbf{Initial population sizes:}

\begin{itemize}
\tightlist
\item
  \emph{initgp}: prior chosen with mean around \(10\)
\item
  \emph{initmsm}: prior chosen with mean around \(10\)
\end{itemize}

See Table 1 for a list of parameters that we are estimating and the
priors used. Note that lower and upper bounds for the priors were used
to keep the posterior distribution at sensible values when using the
BayesianTools R package. If such bounds were not provided negative or
very high values, when low values were expected, could be proposed
during the MCMC.

\rowcolors{2}{gray!6}{white}
\begin{table}

\caption{\label{tab:unnamed-chunk-1}Parameter definition, symbols used in the ODEs, priors with lower and upper bounds}
\centering
\resizebox{\linewidth}{!}{\begin{tabular}[t]{lllll}
\hiderowcolors
\toprule
Parameter & Symbol in R & Prior & Lower & Upper\\
\midrule
\showrowcolors
Spline shape gp0 & gpsp0 & Gamma(3, 3/1.1) & 0.01 & 5\\
Spline shape gp1 & gpsp1 & Gamma(3, 3/1.1) & 0.01 & 5\\
Spline shape gp2 & gpsp2 & Gamma(3, 3/1.1) & 0.01 & 5\\
Spline interval gp & gpsploc & U(1978, 2014) & 1978 & 2014\\
Spline shape msm0 & msmsp0 & Gamma(3, 3/1.1) & 0.01 & 5\\
\addlinespace
Spline shape msm1 & msmsp1 & Gamma(3, 3/1.1) & 0.01 & 5\\
Spline shape msm2 & msmsp2 & Gamma(3, 3/1.1) & 0.01 & 5\\
Spline interval msm & msmsploc & U(1978, 2014) & 1978 & 2014\\
Infectiouness ratio from male to female & maleX & Fixed(1.02) & Not applicable & Not applicable\\
Importation rate & import & Exp(30) & 0 & 0.30\\
\addlinespace
Effective population size of src & srcNe & Exp(20) & 0.0001 & 0.30\\
Probability of infected msm to infect another msm & pmsm2msm & Beta(16, 4) & 0.1 & 1\\
Probability of infected gpf to infect a gpm & pgpf2gpm & Beta(16, 4) & 0.1 & 1\\
Initial number of infected msm & initmsm & Exp(1/10) & 1 & 300\\
Initial number of infected gp & initgp & Exp(1/10) & 1 & 300\\
\bottomrule
\end{tabular}}
\end{table}
\rowcolors{2}{white}{white}

\hypertarget{status-of-analysis}{%
\section{Status of analysis}\label{status-of-analysis}}

\hypertarget{part-1}{%
\subsection{Part 1}\label{part-1}}

I have started estimating the parameters of the model using phydynR. The
first round of analysis, I did not estimate the initial population size
of \emph{gp} and \emph{msm}, and I fixed maleX to 2.0. After meeting
with Erik, he explained that these results are strange given that the
estimated values for the spline function (the 3 shape parameters) should
follow a trend of decreasing rather than increasing. For example:
\(gpsp0 > gpsp1 > gpsp2\)

See attached pdf file (results\_part1.pdf). In this pdf, the density
plots represent several runs that were merged to achieve ESS for each
parameter above 1,000. The black line represents 11 merged independent
MCMC that each were run for 18,000 iterations, while the blue line
represents 10 merged independent MCMC that each were run for 21,000
iterations. The upper and lower bounds for the priors used to generate
these runs were sligtly different for the one in Table 1.

\hypertarget{part-2}{%
\subsection{Part 2}\label{part-2}}

I am currently estimating the parameter values by additionally
estimating the initial population sizes for \emph{gp} and \emph{msm},
and fixing maleX to 1.02.

I still don't have the results for these new analysis, and I am
expecting to have them by next week.

\hypertarget{references}{%
\section*{References}\label{references}}
\addcontentsline{toc}{section}{References}

\hypertarget{refs}{}
\leavevmode\hypertarget{ref-Eilers1996}{}%
Eilers, Paul H. C., and Brian D. Marx. 1996. ``Flexible smoothing with
B-splines and penalties.'' \emph{Statistical Science} 11 (2):89--121.
\url{https://doi.org/10.1214/ss/1038425655}.

\leavevmode\hypertarget{ref-Volz2012}{}%
Volz, Erik M. 2012. ``Complex population dynamics and the coalescent
under neutrality.'' \emph{Genetics} 190 (1):187--201.
\url{https://doi.org/10.1534/genetics.111.134627}.


\end{document}
